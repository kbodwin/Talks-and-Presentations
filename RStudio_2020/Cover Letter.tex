%%%%%%%%%%%%%%%%%%%%%%%%%%%%%%%%%%%%%%%%%
% Professional Formal Letter
% LaTeX Template
% Version 2.0 (12/2/17)
%
% This template originates from:
% http://www.LaTeXTemplates.com
%
% Authors:
% Brian Moses
% Vel (vel@LaTeXTemplates.com)
%
% License:
% CC BY-NC-SA 3.0 (http://creativecommons.org/licenses/by-nc-sa/3.0/)
%
%%%%%%%%%%%%%%%%%%%%%%%%%%%%%%%%%%%%%%%%%

%----------------------------------------------------------------------------------------
%	PACKAGES AND OTHER DOCUMENT CONFIGURATIONS
%----------------------------------------------------------------------------------------

\documentclass[10pt, a4paper]{letter} % Set the font size (10pt, 11pt and 12pt) and paper size (letterpaper, a4paper, etc)
\usepackage{hyperref}
\input{structure.tex} % Include the file that specifies the document structure

%\longindentation=0pt % Un-commenting this line will push the closing "Sincerely," and date to the left of the page

%----------------------------------------------------------------------------------------
%	YOUR INFORMATION
%----------------------------------------------------------------------------------------

\Who{Dr. Kelly Bodwin} % Your name

\Title{, PhD} % Your title, leave blank for no title

\authordetails{
	Department of Statistics\\ % Your department/institution
	Cal Poly State University\\ % Your address
	San Luis Obispo, CA, 93401\\ % Your city, zip code, country, etc
	Email: kbodwin@calpoly.edu\\ % Your email address
	Phone: (408) 482-9935\\ % Your phone number
}

%----------------------------------------------------------------------------------------
%	HEADER CONTENTS
%----------------------------------------------------------------------------------------

\logo{logo.png} % Logo filename, your logo should have square dimensions (i.e. roughly the same width and height), if it does not, you will need to adjust spacing within the HEADER STRUCTURE block in structure.tex (read the comments carefully!)

\headerlineone{CAL POLY STATE UNIVERSITY} % Top header line, leave blank if you only want the bottom line

\headerlinetwo{SAN LUIS OBISPO} % Bottom header line

%----------------------------------------------------------------------------------------

\begin{document}

%----------------------------------------------------------------------------------------
%	TO ADDRESS
%----------------------------------------------------------------------------------------

\begin{letter}{
%	Prof. Jones\\
%	Mathematics Search Committee\\
%	Department of Mathematics\\
%	University of California\\
%	Berkeley, California 12345
}

%----------------------------------------------------------------------------------------
%	LETTER CONTENT
%----------------------------------------------------------------------------------------

\opening{To Whom It May Concern,}

In what has been the easiest decision of my life, I am writing to apply to the 2020 RStudio Internship program.  

I was first introduced to RStudio around 2010, when my classmate Dave Robinson said to me, ``You know, if you used RStudio, you wouldn't have to have 18 scripting windows open at once."  To which I said, ``No thanks, I like it this way."  From that day forward... you know, maybe I should skip ahead a few years, or we'll be here a while.

In seriousness, I am applying to this program because I believe wholeheartedly in RStudio's mission, and I know I am extremely qualified to help advance it.  Since becoming a professor nearly three years ago, my research has progressed linearly in the direction of \href{https://rstudio.csm.calpoly.edu/Normal_Distributions/}{Software Education in Statistics}, particularly with R.  I firmly believe that competency in software like R has the power to unlock a new level of creativity for students working with data.  It is my goal to lower barriers to entry, whether by creating new teaching materials or by developing helper packages or simply by contributing positively to a welcoming community.  In my opinion, RStudio leads the way in these pursuits.

If I spend this summer with RStudio, what I am most excited to contribute to is the development of new primers.  This year, I incorproated the existing tutorials into the first two weeks of my introductory R course, and they proved to be a very valuable asset.  I would love to be involved in extending the coverage of these primers, to include statistical modeling and perhaps also common packages like \{stringr\}.  Because I have taught statistics for years, \href{https://web.calpoly.edu/~kbodwin/r-workshop/}{more recently in R-specific courses}, I am well-placed to help design these tutorials.  I will be able to borrow on a wealth of experience: I know the common misconcpetions, I know which topics that students struggle most with, and I know which \href{https://github.com/kbodwin/decodeR}{examples and activities} have been successful or unsuccessful in my courses.

Another hope I have for this summer is to contribute in a more meaningful way to the development of tidyverse packages.  An embarrassing confession:  I only picked up the tidyverse in 2018, after my first RStudio conference.  Since then, I've had a wonderfully steep learning curve, culminating most recently in \href{https://github.com/kbodwin/flair}{writing my own packages} in the tidy philosophy and contributing successful Pull Requests to \{forcats\} and \{ggplot2\}.  I'm very optimistic about my ability to dive deeper into package development and maintenance, given the time and guidance.

Let me close by acknowledging: Most likely, I personally know the person reading this letter.  As such, I want to reassure you that my love for RStudio and its work will not be diminished if I am not selected for a 2020 internship.  I hope to contribute to these projects and packages regardless of formal association.  Having said that, a summer salary would buy me the free time to truly devote myself to these projects, and the mentorship of the RStudio team will speed up my learning process and help me start contributing sooner and better.  I am unbelievably excited about the chance for me to get involved, and I hope you are too.

\closing{All my best,}

%----------------------------------------------------------------------------------------
%	OPTIONAL FOOTNOTE
%----------------------------------------------------------------------------------------

% Uncomment the 4 lines below to print a footnote with custom text
%\def\thefootnote{}
%\def\footnoterule{\hrule}
%\footnotetext{\hspace*{\fill}{\footnotesize\em Footnote text}}
%\def\thefootnote{\arabic{footnote}}

%----------------------------------------------------------------------------------------

\end{letter}

\end{document}
